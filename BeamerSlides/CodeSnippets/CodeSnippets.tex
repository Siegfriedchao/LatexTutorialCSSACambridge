\documentclass{article}

\usepackage{lipsum}
\usepackage[margin = 1 in, left = 1.5in,includefoot]{geometry}

\usepackage[hidelinks]{hyperref} %Allows for clickable references 


%Tables preamble
\usepackage[none]{hyphenat} % Stops breaking up words in a table 
\usepackage{array}
\newcolumntype{$}{>{\global\let\currentrowstyle\relax}}
\newcolumntype{^}{>{\currentrowstyle}}
\newcommand{\rowstyle}[1]{\gdef\currentrowstyle{#1} #1\ignorespaces}

%Graphics permeable
\usepackage{graphicx} %Allows you to import images
\usepackage{float} %Allows for control of float positions.


%Math preamble
\usepackage{mhchem} % Allows us to write chemistry equations
\usepackage{xfrac} %Allows for slanted fractions


%Bibliography preamble
\usepackage[numbers,sort&compress]{natbib}

% Bullet preamble
\renewcommand{\labelitemi}{$\bullet$}
\renewcommand{\labelitemii}{$\diamond$}
\renewcommand\labelitemiii{$\circ$}

%Header and Footer Stuff
\usepackage{fancyhdr}
\pagestyle{fancy}
\fancyhead{}
\fancyfoot{}
\fancyfoot[R]{ \thepage\ }
\renewcommand{\headrulewidth}{0pt}
\renewcommand{\footrulewidth}{1pt}
%

\begin{document}

\begin{titlepage}
	\begin{center} 
	\line(1,0){300}\\
	[0.25in]
	\huge {\bfseries Daoming \LaTeX{} Notes}\\
	[2mm]
	\line(1,0){200}\\
	[1.5cm]
	\textsc{\LARGE University Of Liverpool}\\
	[0.75cm]
	\textsc{\LARGE Using \LaTeX{} to Write a Simple Report}\\
	[8cm]
	\end{center}
	\begin{flushright}
	\textsc{\large Michael D. \\
	A \LaTeX{} User \\
	\# 1234567890 \\
	January 21, 2016 \\}
	\end{flushright}
\end{titlepage}

%Front matter stuff
\pagenumbering{roman}
\section*{Summary}
\addcontentsline{toc}{section}{\numberline{}Summary}
This is the summary section. This video teaches you how to write a report in \LaTeX{}
\cleardoublepage

\section*{Acknolwdgement}
\addcontentsline{toc}{section}{\numberline{}Acknolwdgement}
Thanks to \LaTeX{}'s creator.
\cleardoublepage

%This is table of contents stuff
\tableofcontents
\thispagestyle{empty}
\cleardoublepage

% List of figures, list of tables
\listoffigures
\addcontentsline{toc}{section}{\numberline{}List of Figures}
\cleardoublepage

\listoftables
\addcontentsline{toc}{section}{\numberline{}List of Tables}
\cleardoublepage

%This is main body stuff
\pagenumbering{arabic}
\setcounter{page}{1}





\section{Introduction}\label{sec:intro}
This is the first line of the report. This report will document on Tobias, the groundhog who lives in my backyard.


Tobias likes to live in a hole in the ground. The text will wrap around properly, don't you worry.


\lipsum[1]

\newpage
\section{Tobia's Lifestyle}
Tobias, although he lives in a hole in the ground, he also likes to climb trees. \cite{ref:zan}

This introduction is found on page \pageref{sec:intro}.
%Here we insert our figure
\begin{figure}[H]
	\centering
	\includegraphics[height=3in]{/Users/daomingdong/Desktop/Owl.jpg}
	\caption[Optional caption]{Real, Local caption}
	\label{fig:Owl}
\end{figure}

Figure \ref{fig:Owl} shows an cute Owl.

\begin{equation}
x = 5
\end{equation}
\subsection{Where Tobias gets food}
Usually groundhogs nibble on greens found on the ground, but not Tobias. He is the exception to this rule.

%Our first table
\begin{table}[H]
	\centering
	\label{tab:tobiastreesightings}
	\caption[This is optional caption, without reference]{Local caption, with reference \cite{ref:two,ref:zan}} 
	\begin{tabular}{ l c r }
		\bfseries{Date} & In tree? & Raining? \\ \hline
		April 26 & Yes & Yes \\
		June 7 & Yes & No \\
		June 20 & Yes & No \\
	\end{tabular}

\end{table}

Table \ref{tab:tobiastreessightthings} logs times I've seen Tobias in the tree.

\begin{table} [H]
	\centering
	\label{tab:ctable}
	\caption{This is our caption for the complicated table}
	\begin{tabular}{$p{0.9in}^p{1.75in}^p{0.9in}^p{1.2in} }
	\rowstyle{\bfseries}Video Number & Subject & Really Long Winded? & Date of filming \\
        1 & Titlepage & Oh yes & Jun 26 \\
        2 & Sections, Margins Page Number & Yes & Jun 26, 14 \\
        3 & Tablescontents Thisistwolongwords & Too much so & Jun 25, 14 \\
        4 & Figures & Not as bad & Jul 2, 14 \\
        5 & Simple Tables & Not really & Jul 2, 14 \\
        6 & Complicated Tables & Probably will be & Jul 2, 14 \\ \hline
        \multicolumn{4}{c}{Speculation} \\ \hline
        7 & Referencing & Hopefully Not & Jul 2, 14 \\
        8 & Appendices & Should not be & Future \\
        9 & List & Not at all & Future \\
        10 & Math & Probably & Future \\
        11 & Different Math & Probably & Future \\
        12 & Why Use \LaTeX{} & Most definitely & Future \\
        \end{tabular}
 \end{table}


\subsubsection{Subsubsection}
This is a subsection.

This is subsection will be used for containing a list!

\begin{itemize}
	\item This is our first line
	\item This is our second line and I am making it longer so that you can see how text wraps around automatically in \LaTeX{}.
	\begin{itemize}
		\item A bullet within a bullet!
			\begin{itemize}
				\item Must go deeper...
			\end{itemize}
		\item Second one too.
	\end{itemize}
	\item Good things come in threes.
	\item [Title] blah blah blah.
	\item[This is a longer title] blah blah blah.
	\begin{enumerate}
		\item Some people
		\item Like lists with numbers instead.
	\end{enumerate}
\end{itemize}

\paragraph{Paragraph heading.}
This can be placed at the beginning of a paragraph
if you want to put a heading on just one paragraph

\cleardoublepage
\bibliographystyle{IEEEtran}
\bibliography{/Users/daomingdong/Documents/SoftwareTraining/LaTex/References/Secondref.bib}
\addcontentsline{toc}{section}{\numberline{}References}
\cleardoublepage

%Appendix starts here
\appendix
\section{Backmatter Words}
This is the appendix.

I can write math in line with text $E=mc^2$ is I wanted to. If I wanted to have math $$Pv = nRT$$ not in a sentence I could do this!

We can use symbols in our math formulas.

$$-\frac{\hbar^2}{2m} \frac{d^2\Psi}{dx^2} = E\Psi$$

Chemistry equation:

\ce{CH_{4(g)} + O_{2(g)} -> CO_{2(g)} +2H_2O_{(l)}}

Fractions

$$d = v_it + 1/2 \cdot at^2$$
$$d = v_it + \frac{1}{2} \cdot at^2$$
$$d = v_it + \sfrac{1}{2} \cdot at^2$$

Brackets:
$$(\frac{1}{2}) \cdot 2 = 1$$
$$\left(\frac{1}{2}\right) \cdot 2 = 1$$ 
\centering
\line(1,0){300}
$$\left|-7 \right| = 7$$

$$x^{2^3}$$


\begin{eqnarray*}
    \sqrt{4} &=& 2 \\
    \sqrt{4} &\neq& 5 \\ 
    \sqrt{4} &<& 5 \\
    \pi &\approx& 3 \\
    \pi &\times& \sqrt{4} < 15
\end{eqnarray*}

\begin{equation} \label{eq:tobias}
x^2 + 7x - 5 = 0
\end{equation}

\begin{equation}\label{eq:second}
x^2 - 9x + 1 = 0
\end{equation}

The first equation is equation \ref{eq:tobias}, which you can try and solve

\end{document}
